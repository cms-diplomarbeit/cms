\section{Einleitung}
\setauthor{Lucas Gastelsberger}
\subsection{Einführung in das Unternehmen und die Software}

CMS Central Management Systems GmbH, mit Sitz in Linz, Österreich, bietet eine spezialisierte Plattform zur zentralisierten Verwaltung von Gebäudetechnik, Energieefffizienz und Betriebssicherheit. Die Software richtet sich an Unternehmen und öffentliche Einrichtungen und unterstützt diese bei der Integration verschiedenster Datenquellen sowie der Automatisierung von Prozessen im Gebäudemanagement.

\subsection{Kernfunktionen und Module}

Die modular aufgebaute Plattform kombiniert Gebäudetechnik, Alarmmanagement, Energiecontrolling und Dokumentation in einem einzigen System. Die wichtigsten Kernfunktionen umfassen:
\begin{itemize}
    \item \textbf{Echtzeitüberwachung \& Energiemanagement}: Überwachung, Analyse und Optimierung des Energieverbrauchs in Echtzeit mit visuellem Ampelsystem zur sofortigen Erkennung ineffizienter oder sicherheitskritischer Zustände
    \item \textbf{Alarm- und Betriebssicherheit}: Priorisierung und Bearbeitung von Alarmen zur Erkennung und raschen Behebung potenzieller Störungen oder Gefahren
    \item \textbf{Dokumentationsmanagement}: Zentrale Verwaltung aller relevanten Betriebs- und Wartungsunterlagen inklusive Reporting und Nachverfolgbarkeit
\end{itemize}

\subsection{Vorteile und Einsatzbereiche}

Durch die zentrale Zusammenführung technischer Gebäudedaten entsteht ein hoher Transparenzgrad, der fundierte Entscheidungen und effizientere Prozesse ermöglicht. Ein Beispiel für effizientere Prozesse sind widerkehrende Abläufe, welche automatisiert werden können. Dies spart Zeit und reduziert Fehlerquellen. Im Hinblick auf gesetzliche Vorgaben, wie etwa die ISO 50001, erfüllt die Plattform alle Voraussetzungen zur Energiedatenerfassung und -analyse. Durch die Verbesserung technischer Abläufe und die Verringerung des Personalaufwands lassen sich zudem die Lebenszykluskosten eines Gebäudes nachhaltig senken.

\subsection{Anwendungsbereiche}

Die Lösung wird in unterschiedlichen Bereichen eingesetzt, darunter Gewerbe- und Industriegebäude, Bildungseinrichtungen und Gesundheitseinrichtungen. In der Industrie liegt der Fokus auf der Überwachung und Optimierung energieintensiver Prozesse, während im Bildungsbereich vor allem eine einfache Verwaltung mehrerer Standorte und die Einhaltung von Nachhaltigkeitszielen im Vordergrund stehen. Im Gesundheitswesen wiederum spielen die Aspekte Zuverlässigkeit und Sicherheitsmanagement eine zentrale Rolle, welche durch das integrierte Alarm- und Dokumentationssystem effektiv abgedeckt werden.

\section{Problemstellung}
\setauthor{Lucas Gastelsberger}

\section{Aufgabenstellung und Anforderungen}
\setauthor{Lucas Gastelsberger}

\section{Ziele}
\setauthor{Lucas Gastelsberger}